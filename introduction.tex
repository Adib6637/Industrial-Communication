\section{Introduction}
Over the last several decades, the increasing power and cost effectiveness of electronic equipment has had an impact on many fields of human endeavor. This is also true for industrial control systems, which are constantly improving. Although digital controllers eventually superseded analogue control, connections to the field were still made via analogue signals. The move to digital systems prompted the creation of new field and controller communication protocols. The most common name for these communication protocols is fieldbus protocols. Recently, digital control systems have begun to encompass networking at all levels of industrial control, as well as the connectivity of commercial and industrial equipment through the use of Ethernet standards.

An industrial communication network serves as the core of any automation system design since it provides a reliable way of data interchange, data controllability, and the ability to connect many devices. The implementation of proprietary digital communication networks in industries over the last decade has resulted in better end-to-end digital signal accuracy and integrity. Industrial communication networks enable smart manufacturing scenarios by enabling real-time communication between machines and control centres. They ensure that supply chain anomalies are identified and prevented by monitoring processes and activities. Predictive maintenance is a great example of how the Industrial Internet of Things may increase efficiency and cost optimization by connecting and analysing data.

This paper serves as an introduction to industrial communication networks. Industrial networking concerns itself with the implementation of communications protocols between field equipment, digital controllers, various software suites and also to external systems. To assist the reader, the communication network that is widely used specially in industrial field will be explain generally so that the reader can have humongous imagination and deep understanding regarding this topic. The chosen communication network which is CAN will be thoroughly discussed and contrast with those other networks. Many aspects of the operation and philosophy of industrial networks has evolved over a significant period of time and as such a history of the field is provided. The operation of modern control networks is examined, and some popular protocols are described. Any machine with sensors, for example, may collect critical information about its status and interactions with other equipment and operators.

