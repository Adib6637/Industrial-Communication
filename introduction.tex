\section{Introduction}
The rising power and cost effectiveness of electronic devices has affected all aspects of human endeavour during the last several decades. This also not exceptional to industrial control systems which are always improvise progressively. Control of industrial and process facilities was first done mechanically in the early 20th century, either manually with analogue devices or through the use of hydraulic controllers. Mechanical control systems were superseded by electronic control loops that used transducers, relays, and hard-wired control circuits as discrete electronics became more prevalent now a days. These systems were enormous and took up a lot of area, frequently necessitating many kilometres of wire, starting from one point to another point, as example the connection from the field itself to the control circuits. The functionality of many analogue control loops may be reproduced by a single digital controller thanks to the introduction of integrated circuits and microprocessors.

We know that at the beginning phase, a digital computer was for the first time being applied as a digital controller. As time goes by, Digital controllers gradually replaced analogue control, however connection with the field was still done using analogue signals. The transition to digital systems necessitated the development of new communication protocols for the field as well as between controllers. Fieldbus protocols are the most frequent name for these communication protocols. Recently, digital control systems began to include networking at all levels of industrial control, as well as the interconnection of commercial and industrial equipment utilizing Ethernet standards. This has resulted in a networking environment that, on the surface, looks to be comparable to traditional networks but has fundamentally different requirements.

An industrial communication network is the foundation of any automation system design since it provides a strong method of data interchange, data controllability, and the flexibility to link numerous devices. Over the last decade, the deployment of proprietary digital communication networks in industries has resulted in improved end-to-end digital signal correctness and integrity. With the help of real-time communication among machines and control centers, industrial communication networks power smart manufacturing scenarios. They guarantee that supply chain abnormalities may be evaluated and avoided by monitoring procedures and operations. Predictive maintenance is an excellent illustration of how the Industrial Internet of Things may improve efficiency and cost optimization through connection and data analytics. For example, any machine with sensors may collect vital information about its state and interactions with other equipment and operators. The network transmits this data to processing centers, where it is compared to historical and statistical data to find trends, recurrences, and exceptions. They can also build links between certain inputs and any breakdowns or inefficiencies. Intelligent gateways provide smooth interaction between several communication protocols that are structured in separate subnetworks.

This paper serves as an introduction to industrial communication networks. Industrial networking concerns itself with the implementation of communications protocols between field equipment, digital controllers, various software suites and also to external systems. To assist the reader, the communication network that is widely used specially in industrial field will be explain generally so that the reader can have humongous imagination and  deep understanding regarding this topic. The chosen communication network which is CAN will be thoroughly discussed and contrast with those other networks. Many aspects of the operation and philosophy of industrial networks has evolved over a significant period of time and as such a history of the field is provided. The operation of modern control networks is examined and some popular protocols are described