
\section{State Of The Art}
The CAN bus has played an essential role in the industry since its inception in 1987. Many companies, notably those in the automotive industry, are concentrating on CAN development. BMW, for example, first employed CAN bus technology in 1995. They used a star topology network with five electronic control units in their 7 Series automobiles. They then expanded the concept by integrating a second network for body electronics. This allowed two separate CAN-BUS networks to be linked via gateway connections. CAN systems gain trust in the automotive sector through protocol standardization and security testing to ensure system compatibility. [4] The CAN bus has established a name for itself not just in the automobile industry, but also in the medical arena. Philips Medical Systems, for example, was one of the first medical equipment manufacturers to employ the CAN bus for internal networking of their X-ray machines. [5] It's probable that vending machines will be created in the near future. As observed, a Chinese start-up company developed a prototype and deployed a CAN network. The prototype has the ability to make coffee. In other cases, it might be supplied with vending machines that sell vegetables and other meal supplies. Shuttle services using self-driving vehicles and autonomous freight will also be possible. A CAN-compliant vehicle control unit is installed in the vehicle. It controls the lights, horn, windshield wiper, handles, accelerator, and other functions. The electric steering unit also supports CAN connectivity. [6]