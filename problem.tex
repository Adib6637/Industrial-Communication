
	Wired communication protocol always being used in many application because of its reliability in term of the connection and it efficiency especially  in transferring long sequence of data.
	
	Today, there is many protocol that support weird communication such as UART, SPI and I2C as written in the previous subtopic. each protocol have their on advantage to solve many problem in a curtain application domain. For instance, SPI support many salve to be controlled by one master that almost impossible to be implemented using UART. This kind of protocol use atleast 4 wire to connect between one master and one slave. By using this protocol, increasing number of slave means that the used number of wire for the communication also will be increased. This is a huge problem for a system that use a lot of distinct sensors and actuators smartphone, computer and machine in production line. Here is where the I2C play a big role because with the I2C, only two wire are need for the communication between a master and their slaves regardless of the number of the salve.
	
	Another problem that arrive when using I2C in a more rugged application is their reliability and efficiency. Imagine a main controller for a car system that is located at the front of ther car want to communicate withe the brake lamp at the back of the car, where the length of the used wire could be 5 meter.. Since the I2C protocol only support for low voltage application, with that long, it is possible that output from the controller will not be well received by the brake lamp system. If the output is well receive it cloud be very slow becouse with that long, we only allowed to use very low frequency to make the communication works. But to make it work is not only the purpose. In case on emergency brake, it is important for the car behind to that this car a breaking through the brake lamp. If the signal is to slow this can cause a catastrophic event.
	
	To solve this problem which is to reduce the number of used wire, reliable and efficient communication protocol for rugged application, Controller Area Network protocol or CAN protocol should be used. 
	
	We will explain more bout this protocol in the next subtopic followed by the application to make user more understand and able to implement this kind of protocol.



