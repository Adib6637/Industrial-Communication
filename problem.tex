
\section{Problem}

Many electronic components make up today's automobile, and they all communicate with one another. For example, a system that receives human input, such as a pedal brake, will connect with the brake lamp system. At least two wires may be required for communication. In addition, other systems, such as the safety belt system and the door lock system, also  need the user's input. Means that more and more cable will be added to this communication environment. This might get a lot messier and more difficult to maintain.

The core principle of the approach is to use a bus system. The bus system should meet our primary criteria of using less cable across the communication system. I2C is a common bus system protocol that requires only two wires for the whole communication system. After experimenting with I2C, we discovered that it is not a viable solution since communication becomes unreliable when the cable length exceeds one meter. If we choose a very low frequency, which indicates a low data rate transmission, the communication becomes stable over a long distance.

When a driver has to apply an emergency brake, communicating at a low frequency can cause the brake lamp system to respond slowly, causing the car behind to be unable to respond quickly. As a result, you'll have a disastrous outcome. That is, we need to enhance our communication protocol requirements by adding additional parameters such as support for high frequency or high data rate transfer across long distances.

To meet all of these requirements, we'll need to conduct more study on a protocol that can fix the problem. As a result, the optimum solution for this problem or demand is the Controller Area Network. In the next part, we'll go through the protocol in in detail.