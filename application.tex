
\section{Application of CAN Protocol}

Now we will go through our project in  developing a prototype subsystem in a car that use CAN protocol as its communication standard.  For that, we use the problem as state in problem section as our main  reference for our prototype.

After we analyse the problem, we refine our requirement in requirement diagram as shown in Figure \ref{fig:requirement_diagram}. As we mention before, the protocol should work for long distance at high frequency. 

\begin{figure}[h]
    \centering
    \includegraphics[width=0.5\textwidth]{requirement_diagram}
    \caption{Requirement Diagram}
    \label{fig:requirement_diagram}
\end{figure}

As we explain in the problem section, we choose Controller area network because it supports high data rate transfer as stated in the pervious section. Next, we will explain how we model our solution base on the requirement.

\subsection{Application Model}

Our aim at this stage, which is modelling our solution, is to have a concrete system architecture including communication system where we will implement the communication protocol, so that it will be clear for us to do the implementation. The steps in this stage will be discussed in this section.

After analysing the scenario, requirements and protocol that we want to use, we create a use case diagram to have a clear boundary of the system as shown in Figure \ref{fig:use_case_dagram}.

\begin{figure}[h]
    \centering
    \includegraphics[width=0.5\textwidth]{use_case_dagram}
    \caption{Use Case Diagram}
    \label{fig:use_case_dagram}
\end{figure}

Next, we extend each use case using activity diagrams to understand their behaviour as pictured in Figure \ref{fig:AD_pedalpush}, and \ref{AD_pedalrelease}. later we use all this diagram to make a sequence diagram to illustrate how object in the subsystem interact with each other as shown in Figure \ref{fig:sequence_diagram}.

\begin{figure}[h]
    \centering
    \includegraphics[width=0.5\textwidth]{AD_pedalpush}
    \caption{Activity diagram when the pedal pedal pushed}
    \label{fig:fig:AD_pedalpush}
\end{figure}
\begin{figure}[h]
    \centering
    \includegraphics[width=0.5\textwidth]{AD_pedalrelease}
    \caption{Activity diagram when the is pedal released}
    \label{fig:AD_pedalrelease}
\end{figure}


\begin{figure}[h]
    \centering
    \includegraphics[width=0.5\textwidth]{sequence_diagram}
    \caption{Sequence Diagram}
    \label{fig:sequence_diagram}
\end{figure}

Eventually, by analysing all the pervious diagram,  we are able to refine our system architecture as shown in Figure \ref{fig:system_architecture}.  This indicates that we are ready for implementation stage, which will be the focus of the following subsection.

\begin{figure}[h]
    \centering
    \includegraphics[width=0.5\textwidth]{system_architecture}
    \caption{System Architecture}
    \label{fig:system_architecture}
\end{figure}

\subsection{Implementation}
For the implementation we divide it into two parts which is hardware and software part. We will first explain the hardware part followed by software part.

\subsubsection{Hardware}

For the hardware setup we will use Arduino UNO with CAN module as shown in Figure \ref{fig:MCP2521} to represent each subsystem as drawn in the system architecture that we present before in Figure \ref{fig:system_architecture} which are pedal system, database system and brake lamp system as shown in figure \ref{fig:subsystem_pedal}, \ref{fig:subsystem_database},\ref{fig:subsystem_lamp}. The CAN module that we are used is MCP2515. For the connection with between the Arduino and the CAN module, we use SPI communication protocol. Considering that the protocol should work for long distance, we also use ca. 1.5 meter for the connection between subsystem. After we have done the setup as illustrate in Figure \ref{fig:hardware_setup} we proceed to the next step which is the software implementation.

\begin{figure}[h]
    \centering
    \includegraphics[width=0.5\textwidth]{MCP2521}
    \caption{MCP2521}
    \label{fig:MCP2521}
\end{figure}

\begin{figure}[h]
    \centering
    \includegraphics[width=0.5\textwidth]{system_architecture}
    \caption{System architecture}
    \label{fig:System architecture}
\end{figure}
\begin{figure}[h]
    \centering
    \includegraphics[width=0.5\textwidth]{subsystem_pedal}
    \caption{Pedal subsystem}
    \label{fig:subsystem_pedal}
\end{figure}
\begin{figure}[h]
    \centering
    \includegraphics[width=0.5\textwidth]{subsystem_database}
    \caption{Database subsystem}
    \label{fig:subsystem_database}
\end{figure}
\begin{figure}[h]
    \centering
    \includegraphics[width=0.5\textwidth]{subsystem_lamp}
    \caption{Brake lamp subsystem}
    \label{fig:subsystem_lamp}
\end{figure}
\begin{figure}[h]
    \centering
    \includegraphics[width=0.5\textwidth]{hardware_setup}
    \caption{Hardware setup}
    \label{fig:hardware_setup}
\end{figure}

\subsubsection{Software}

Since we use Arduino as our subsystem platform, we use Arduino IDE to implement our software. To make the communication works as we had planned, we use CAN library provided by Arduino. In proving that It could work at high frequency, we set the data rate transfer to 1MBps. The code for each subsystem can be found in \url{https://www.instagram.com/_sheikh_adib/?hl=en}.